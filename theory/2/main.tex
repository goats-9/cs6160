\documentclass[journal,12pt,twocolumn]{IEEEtran}
\usepackage{setspace}
\usepackage{gensymb}
\singlespacing
\usepackage[cmex10]{amsmath}
\usepackage{amsthm}
\usepackage{mathrsfs}
\usepackage{txfonts}
\usepackage{stfloats}
\usepackage{bm}
\usepackage{cite}
\usepackage{cases}
\usepackage{subfig}
\usepackage{longtable}
\usepackage{multirow}
\usepackage{enumitem}
\usepackage{mathtools}
\usepackage{tikz}
\usepackage{circuitikz}
\usepackage{verbatim}
\usepackage[breaklinks=true]{hyperref}
\usepackage{tkz-euclide} % loads  TikZ and tkz-base
\usepackage{listings}
\usepackage{color}    
\usepackage{array}    
\usepackage{longtable}
\usepackage{calc}     
\usepackage{multirow} 
\usepackage{hhline}   
\usepackage{ifthen}   
\usepackage{lscape}     
\usepackage{chngcntr}
\usepackage{algorithm}
\usepackage{algpseudocodex}
\DeclareMathOperator*{\Res}{Res}
\renewcommand\thesection{\arabic{section}}
\renewcommand\thesubsection{\thesection.\arabic{subsection}}
\renewcommand\thesubsubsection{\thesubsection.\arabic{subsubsection}}

\renewcommand\thesectiondis{\arabic{section}}
\renewcommand\thesubsectiondis{\thesectiondis.\arabic{subsection}}
\renewcommand\thesubsubsectiondis{\thesubsectiondis.\arabic{subsubsection}}
\renewcommand\thetable{\arabic{table}}
% correct bad hyphenation here
\hyphenation{op-tical net-works semi-conduc-tor}
\def\inputGnumericTable{}                                 %%

\lstset{
%language=C,
frame=single, 
breaklines=true,
columns=fullflexible,
literate=
{-}{$\rightarrow{}$}{1},
}
%\lstset{
%language=tex,
%frame=single, 
%breaklines=true
%}

\DeclareMathOperator*{\argmax}{arg\,max}
\DeclareMathOperator*{\argmin}{arg\,min}
\begin{document}
\newtheorem{theorem}{Theorem}[section]
\newtheorem{problem}{Problem}
\newtheorem{proposition}{Proposition}[section]
\newtheorem{lemma}{Lemma}[section]
\newtheorem{corollary}[theorem]{Corollary}
\newtheorem{example}{Example}[section]
\newtheorem{definition}[problem]{Definition}
\newcommand{\BEQA}{\begin{eqnarray}}
\newcommand{\EEQA}{\end{eqnarray}}
\newcommand{\define}{\stackrel{\triangle}{=}}
\bibliographystyle{IEEEtran}
\providecommand{\mbf}{\mathbf}
\providecommand{\pr}[1]{\ensuremath{\Pr\left(#1\right)}}
\providecommand{\qfunc}[1]{\ensuremath{Q\left(#1\right)}}
\providecommand{\sbrak}[1]{\ensuremath{{}\left[#1\right]}}
\providecommand{\lsbrak}[1]{\ensuremath{{}\left[#1\right.}}
\providecommand{\rsbrak}[1]{\ensuremath{{}\left.#1\right]}}
\providecommand{\brak}[1]{\ensuremath{\left(#1\right)}}
\providecommand{\lbrak}[1]{\ensuremath{\left(#1\right.}}
\providecommand{\rbrak}[1]{\ensuremath{\left.#1\right)}}
\providecommand{\cbrak}[1]{\ensuremath{\left\{#1\right\}}}
\providecommand{\lcbrak}[1]{\ensuremath{\left\{#1\right.}}
\providecommand{\rcbrak}[1]{\ensuremath{\left.#1\right\}}}
\theoremstyle{remark}
\newtheorem{rem}{Remark}
\newcommand{\sgn}{\mathop{\mathrm{sgn}}}
\providecommand{\abs}[1]{\left\vert#1\right\vert}
\providecommand{\res}[1]{\Res\displaylimits_{#1}} 
\providecommand{\norm}[1]{\left\lVert#1\right\rVert}
\providecommand{\mtx}[1]{\mathbf{#1}}
\providecommand{\mean}[1]{E\left[ #1 \right]}   
\providecommand{\fourier}{\overset{\mathcal{F}}{ \rightleftharpoons}}
\providecommand{\system}[1]{\overset{\mathcal{#1}}{ \longleftrightarrow}}
\newcommand{\solution}{\noindent \textbf{Solution: }}
\newcommand{\cosec}{\,\text{cosec}\,}
\providecommand{\dec}[2]{\ensuremath{\overset{#1}{\underset{#2}{\gtrless}}}}
\newcommand{\myvec}[1]{\ensuremath{\begin{pmatrix}#1\end{pmatrix}}}
\newcommand{\mydet}[1]{\ensuremath{\begin{vmatrix}#1\end{vmatrix}}}
\renewcommand{\vec}[1]{\boldsymbol{\mathbf{#1}}}
\def\putbox#1#2#3{\makebox[0in][l]{\makebox[#1][l]{}\raisebox{\baselineskip}[0in][0in]{\raisebox{#2}[0in][0in]{#3}}}}
     \def\rightbox#1{\makebox[0in][r]{#1}}
     \def\centbox#1{\makebox[0in]{#1}}
     \def\topbox#1{\raisebox{-\baselineskip}[0in][0in]{#1}}
     \def\midbox#1{\raisebox{-0.5\baselineskip}[0in][0in]{#1}}

\vspace{3cm}
\title{CS6160 Assignment 2}
\author{Gautam Singh\\CS21BTECH11018}
\maketitle
\bigskip

\begin{enumerate}
    \item Note that
    
    \begin{align}
        B\brak{g^\alpha} \triangleq g^{\alpha^2} &= g^{\alpha^2 + \alpha - \alpha} \\
                                                 &= \frac{g^{\alpha\brak{\alpha+1}}}{g^\alpha} \\
                                                 &= \frac{A\brak{g^{\frac{1}{\alpha\brak{\alpha+1}}}}}{g^\alpha} \\
                                                 &= \frac{1}{g^\alpha}A\brak{\frac{g^{\frac{1}{\alpha}}}{g^{\frac{1}{\alpha+1}}}} \\
                                                 &= \frac{1}{g^\alpha}A\brak{\frac{A\brak{g^\alpha}}{A\brak{g^{\alpha+1}}}}
                                                 \label{eq:B-def}
    \end{align}
    
    can be computed since \(g, p, q, g^\alpha\) are known.

    From \eqref{eq:B-def}, we compute

    \begin{align}
        C\brak{g^\alpha, g^\beta} &\triangleq \frac{B\brak{g^{\alpha+\beta}}}{B\brak{g^\alpha}B\brak{g^\beta}} \\
                                  &= g^{\brak{\alpha+\beta}^2-\alpha^2-\beta^2} = g^{2\alpha\beta}.
                                  \label{eq:C-def}
    \end{align}

    Finally, we compute using \eqref{eq:C-def},

    \begin{align}
        F\brak{g^\alpha, g^\beta} &\triangleq A\brak{\brak{A\brak{C\brak{g^\alpha,g^\beta}}}^2} \\
                                  &= A\brak{\brak{A\brak{g^{2\alpha\beta}}}^2} \\
                                  &= A\brak{g^{\frac{1}{\alpha\beta}}} = g^{\alpha\beta}.
                                  \label{eq:F-sol}
    \end{align}

    \item Using the \emph{Chinese Remainder Theorem}, we can solve the system of
    congruences

    \begin{align}
        r^3 &\equiv c_1 \brak{\bmod\ N_1} \\
        r^3 &\equiv c_2 \brak{\bmod\ N_2} \\
        r^3 &\equiv c_3 \brak{\bmod\ N_3}
        \label{eq:crt-equiv}
    \end{align}

    modulo \(N_1N_2N_3\), since the \(N_i\) are pairwise co-prime (if this was
    not the case, then the RSA public keys could be factored out by taking a
    GCD). Further, since \(r \in \mathbb{Z}_{N_1}\), we have \(r < N_1 < N_2 <
    N_3\), we have \(r^3 < N_1N_2N_3\), thus our unique solution is indeed equal
    to \(r^3\). Taking a cube root of this solution gives us \(r\).

    Having found \(r\), we can now compute \(H\brak{r}\) and recover \(H\brak{r}
    \oplus \brak{H\brak{r} \oplus m} = m\).

    \item 
    \begin{enumerate}
        \item Given \(\brak{i, \mathrm{Sign}\brak{i}} = \brak{i,
        f^{\brak{n-i}}\brak{x}}\), the receiver can verify the signature by
        computing
        \begin{equation}
            f^{\brak{i}}\brak{f^{\brak{n-i}}\brak{x}} = f^{\brak{n}}\brak{x}
            \label{eq:pubkey-recov}
        \end{equation}
        and verifying that it is equal to the public key. If not, then the
        signature is invalid for the given \(i\).

        \item Notice that when \(f^{\brak{n-i}}\brak{x}\) is computed, from the
        above, we also compute the values of \(f^{\brak{j}}\brak{x}\) for \(n -
        i \le j \le n\). Setting \(i = n\), we see that we obtain
        \(f^{\brak{i}}\brak{x}\) for all \(1 \le i \le n\), hence we can forge
        every message in \(\mathcal{M}\) after knowing the tag for message
        \(n\).
    \end{enumerate}

    \item 
    \begin{enumerate}
        \item For the scheme to be one-time secure, \(f\) must be
        \emph{subset-free}, that is, there do not exist \(m_1 \neq m_2\) such
        that \(f\brak{m_1} \subset f\brak{m_2}\). Since \(f\) maps messages to
        subsets of size \(k\), it follows that no two images can be subsets of
        each other. Thus, for being subset-free, we must have
        \begin{equation}
            2^n \le \binom{2t}{k}.
            \label{eq:subset-free-cond}
        \end{equation}
        Hence, the values of \(k\) are 
        \begin{equation}
            S_k = \cbrak{k: k \in \cbrak{0,1,\ldots,2t},\ 2^n \le \binom{2t}{k}}.
            \label{eq:Sk-ans}
        \end{equation}

        \item From \eqref{eq:subset-free-cond}, we have
        \begin{equation}
            2^n \le \binom{2t}{k} \le \binom{2t}{t}.
            \label{eq:m-bound}
        \end{equation}
        Asymptotically, for large \(t\),
        \begin{align}
            2^n &\le \frac{4^t}{\sqrt{\pi t}} \\
            \implies n &\le 2t - \frac{1}{2}\log\brak{\pi t} \\
            &= \mathcal{O}\brak{t - \log{t}}.
            \label{eq:n-asymptote}
        \end{align}
    \end{enumerate}
\end{enumerate}

\end{document}

